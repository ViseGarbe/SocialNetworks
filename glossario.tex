\newglossaryentry{corpus}{
    name=corpus,
    description={In ambito linguistico e testuale, un corpus è un insieme strutturato e selezionato di testi raccolti secondo criteri specifici, utilizzato per l’analisi quantitativa e qualitativa di fenomeni linguistici, stilistici o discorsivi. Un corpus può essere costituito da testi scritti, trascrizioni di parlato o materiali multimediali, a seconda degli obiettivi della ricerca}
}

\newglossaryentry{bias-linguistici}{
    name=bias linguistici,
    description={Declinazioni sistematiche e non neutre nel linguaggio utilizzato per descrivere eventi, persone o fenomeni, che riflettono e talvolta rafforzano pregiudizi culturali, politici o ideologici. I bias linguistici si manifestano attraverso scelte lessicali, sintattiche o stilistiche che possono influenzare la percezione e l'interpretazione da parte del destinatario}
}

\newglossaryentry{survey-internazionali}{
    name=survey-internazionali,
    description={Indagini statistiche condotte su scala globale o sovranazionale, finalizzate alla raccolta sistematica di dati comparabili tra differenti paesi o aree geografiche. Tali rilevazioni utilizzano metodologie standardizzate per analizzare opinioni, atteggiamenti, comportamenti o caratteristiche socio-demografiche, consentendo confronti empiricamente fondati a livello internazionale}
}

\newglossaryentry{framing}{
    name=framing,
    description={Il framing (o effetto framing) si riferisce al modo in cui una situazione o un'informazione sono presentati, e come questo influenza la percezione e l'interpretazione che ne fa il pubblico. In sostanza, è come se un'immagine fosse incorniciata: la cornice, ovvero il modo in cui l'immagine è presentata, può cambiare significativamente la nostra percezione della stessa, anche se il contenuto rimane invariato.}
}

\newglossaryentry{agenda-setting}{
    name=agenda setting,
    description={L'agenda-setting, nel campo della comunicazione, è una teoria che spiega come i media siano in grado di influenzare la percezione del pubblico riguardo agli argomenti più significativi. Essi riescono a "definire l'agenda" del dibattito pubblico attraverso la selezione delle notizie ritenute rilevanti e attribuendo a queste un determinato livello di importanza.}
}

\newglossaryentry{frame-interpretativi}{
    name=frame-interpretativi,
    description={I frame interpretativi rappresentano strumenti che condizionano la nostra comprensione e interpretazione del mondo. Essi funzionano analogamente a "cornici" concettuali, facilitando la selezione, l'organizzazione e l'attribuzione di significato agli eventi e alle informazioni che assimilamo, modellando in tal modo la nostra percezione complessiva della realtà.}
}