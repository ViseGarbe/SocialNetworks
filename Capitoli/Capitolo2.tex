\chapter{La politicizzazione del linguaggio giornalistico}

La relazione tra giornalismo e politica è estremamente complessa, un continuo equilibrio tra collaborazione e conflitto, in quanto i politici necessitano dei media per la divulgazione delle loro azioni, ma volendone controllare il racconto. D'altra parte, i giornalisti, seppur dipendendo da tali fonti politiche, desiderano indipendenza e libertà di espressione critica \cite{giornalismo&politica}. Secondo Gianpietro Mazzoleni \cite{gpMazzoleni}, i media svolgono tre funzioni principali: informare, interpretare e persuadere.\
Le prime due sono a "disposizione degli utenti": i giornalisti rappresentano la popolazione svolgendo la funzione di mediatori tra la politica e i cittadini, offrendo un'informazione oggettiva e interpretando le notizie. La terza funzione – quella persuasiva – invece solleva interrogativi etici e politici, poiché rischia di trasformare i media in strumenti di influenza ideologica piuttosto che garanti di un'informazione libera ed equilibrata. \\
Uno dei principali problemi in questo contesto riguarda la stretta relazione sociale tra giornalisti e politici. Fabio Martini \cite{fMartini} sottolinea come i giornalisti politici vivano immersi nello stesso ambiente dei rappresentanti della politica: frequentano gli stessi ristoranti, partecipano agli stessi circoli sociali e, in molti casi, i loro figli frequentano le medesime scuole. Questa vicinanza rischia di minare l’indipendenza del racconto giornalistico, creando dinamiche di complicità, autocensura o persino collusione tra chi descrive la politica e chi la pratica. \\
Martini mette in luce come simili meccanismi relazionali emergano anche in altri ambiti professionali: tra giornalisti giudiziari e magistrati, protagonisti del mondo dello spettacolo e registi, oppure tra giornalisti economici e imprenditori. Tuttavia, è soprattutto nel rapporto tra stampa politica e classe dirigente che queste connessioni risultano più delicate e potenzialmente dannose per la trasparenza del sistema democratico. In questo scenario, il linguaggio giornalistico tende ad essere influenzato dalla politicizzazione, con effetti negativi sulla neutralità del discorso. Questo fenomeno favorisce la creazione di narrazioni orientate agli interessi delle élite politiche coinvolte. \\
La politicizzazione non si limita alla scelta dei toni e dei termini, ma si estende anche alla selezione delle informazioni, al modo in cui gli eventi vengono inquadrati (\gls{framing}) e alla definizione delle priorità nell’agenda mediatica. Questi processi contribuiscono alla costruzione di una realtà politica mediata dai mezzi di comunicazione, spesso distante dalla sua dimensione oggettiva. \\

\section{Definizioni ed Implicazioni}
Per analizzare in modo approfondito la questione relativa alla politicizzazione del linguaggio giornalistico, è indispensabile definire con precisione il significato di questa espressione. La politicizzazione può essere descritta come il processo mediante il quale il linguaggio e le pratiche del giornalismo subiscono un'influenza diretta da logiche, interessi o retoriche tipiche del discorso politico, arrivando così a perdere, parzialmente o totalmente, la loro originaria funzione di mediazione neutrale e oggettiva dell'informazione. \\
Questo fenomeno si manifesta non solo attraverso l’adozione di un lessico orientato ideologicamente, ma anche nelle modalità di selezione delle fonti, nella strutturazione gerarchica delle notizie, nei meccanismi di inquadramento narrativo (\gls{framing}) e nelle strategie volte a determinare le priorità dell’agenda informativa (\gls{agenda-setting}). In una simile prospettiva, il giornalismo rischia di mutare la propria natura trasformandosi in un mezzo di propaganda, abdicando al suo ruolo critico e di servizio pubblico per favorire narrazioni funzionali ai detentori del potere. \\
Un’informazione politicizzata contribuisce alla polarizzazione dell’opinione pubblica, all’irrigidimento delle divisioni ideologiche ed alla crescente sfiducia nei confronti dei mezzi di comunicazione. Inoltre, essa compromette il principio della libertà di stampa, subordinando la pratica giornalistica ad interessi esterni. \\
Sul piano etico, la politicizzazione richiede una riflessione approfondita sul ruolo del giornalista come attore sociale responsabile. Qualora il linguaggio utilizzato da tali professionisti risulti piegato alle logiche del potere, emergono interrogativi cruciali riguardanti la veridicità, l’imparzialità e la responsabilità civica della comunicazione giornalistica. \\
Da una prospettiva sociologica, è inoltre essenziale esaminare in che modo la politicizzazione influenzi la percezione collettiva della realtà politica. I mezzi di comunicazione non si limitano infatti a trasmettere i fatti: essi li selezionano, li interpretano e li rappresentano secondo schemi narrativi che partecipano attivamente alla costruzione di un determinato immaginario politico. \\
Pertanto, la politicizzazione del linguaggio giornalistico non si configura soltanto come una problematica interna alla professione, bensì come un elemento che contribuisce in maniera determinante a plasmare il contesto politico e culturale di una società.

\section{Il ruolo della retorica e della semantica}
Nel processo di politicizzazione del linguaggio giornalistico, la retorica e la semantica rivestono un ruolo fondamentale, si configurano come strumenti chiave nella costruzione di una rappresentazione orientata della realtà politica. Attraverso approfondite scelte lessicali, strutture narrative definite e l'utilizzo di figure retoriche, i media contribuiscono non solo a descrivere, ma anche a plasmare l'immagine che il pubblico si forma degli eventi. \\
La retorica, considerata "l'arte di persuadere per mezzo del linguaggio", viene utilizzata strategicamente per influenzare l'opinione pubblica sia in maniera esplicita che implicita. Definire, ad esempio, una manifestazione come “protesta pacifica” anziché “rivolta violenta” non rappresenta una semplice variazione stilistica, ma costituisce un atto semantico e politico che contribuisce a modellare la percezione dei fatti da parte del pubblico. \\
La dimensione semantica si manifesta soprattutto nella selezione del vocabolario e nella costruzione dei concetti. Parole come riforma, tagli, sicurezza, popolo, élite o emergenza non possiedono neutralità oggettiva, veicolano valori e orientamenti ideologici sottesi che riflettono la prospettiva dell’autore ed, al contempo, influenzano la possibile interpretazione. In tal modo, il linguaggio giornalistico gioca un ruolo estremamente cruciale nella normalizzazione di determinati punti di vista, presentando come inevitabili o necessarie scelte politiche che potrebbero essere soggette a maggiore scrutinio critico. \\
Un aspetto particolarmente rilevante di questa dinamica è la funzione performativa del linguaggio. I media non si limitano alla semplice descrizione degli eventi; al contrario, partecipano attivamente alla costruzione della realtà attraverso un uso, teoricamente, consapevole e strategico delle parole. La selezione di termini e di formule espressive genera effetti cognitivi ed emotivi sul pubblico, il che solleva interrogativi in merito alla responsabilità etica del giornalista. \\
La retorica giornalistica si rivela cruciale nella costruzione di \gls{frame-interpretativi} che orientano la comprensione collettiva dei fenomeni politici. Raccontare crisi, conflitti, eroi o nemici secondo schemi narrativi ricorrenti semplifica la complessità degli eventi e facilita la mobilitazione emotiva dell'audience. Pertanto, la semantica e la retorica si configurano come strumenti essenziali nel processo di politicizzazione del linguaggio mediatico, con un impatto profondo sia sulla percezione della realtà che sul discorso pubblico complessivo. \\