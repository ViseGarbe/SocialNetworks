\chapter{L’attore principale: Donald J. Trump}

\section{Profilo biografico e politico}

Donald John Trump nasce il 14 giugno 1946 a New York, figlio di Fred Trump, imprenditore immobiliare, e Mary Anne MacLeod, immigrata scozzese. Cresciuto nel Queens, frequenta la New York Military Academy e si laurea in economia presso la Wharton School dell’Università della Pennsylvania nel 1968. Dopo gli studi, entra nell’azienda di famiglia, la Trump Organization, espandendo le attività immobiliari in progetti di grande visibilità come la Trump Tower a Manhattan, hotel, casinò e campi da golf, consolidando la sua immagine di imprenditore di successo \cite{blair2001}. \\
Negli anni Ottanta e Novanta, Trump diventa una figura pubblica nota non solo per le sue attività imprenditoriali, ma anche per il suo stile di vita sfarzoso e la presenza costante nei media. La sua popolarità cresce ulteriormente grazie al reality show \emph{The Apprentice}, che lo trasforma in un personaggio televisivo di rilievo nazionale \cite{mcintosh2016}. \\
L’ingresso formale in politica avviene nel 2015, quando annuncia la candidatura alle primarie repubblicane per le elezioni presidenziali del 2016. La sua campagna si distingue per un linguaggio diretto, spesso provocatorio, e per l’uso innovativo dei social media, in particolare Twitter, come strumento di comunicazione politica \cite{ott2017}. I temi centrali della sua piattaforma includono il rafforzamento dei confini, la revisione degli accordi commerciali, la riduzione delle tasse e una politica estera improntata all’“America First”. Nonostante lo scetticismo iniziale, Trump vince le primarie e, nel novembre 2016, viene eletto 45º Presidente degli Stati Uniti, sovrastando Hillary Clinton. \\
Il suo mandato (2017–2021) è segnato da una forte polarizzazione politica e sociale, da una comunicazione fuori dagli schemi tradizionali e da decisioni controverse sia in politica interna che estera. Tra le iniziative più rilevanti si ricordano la riforma fiscale del 2017, la nomina di tre giudici alla Corte Suprema, il ritiro da accordi internazionali, come il trattato di Parigi sul clima e l’accordo sul nucleare iraniano, e la gestione della pandemia di COVID-19 \cite{baker2020}. \\
Dopo la sconfitta alle elezioni del 2020 contro Joe Biden, Trump ha continuato a esercitare una forte influenza sul Partito Repubblicano e sull’elettorato conservatore, alimentando teorie di frode elettorale e mantenendo un ruolo centrale nel dibattito politico statunitense \cite{graham2021}. La sua figura è rimasta estremamente divisiva, rappresentando per alcuni la difesa dei valori tradizionali americani e, per altri, un simbolo di populismo e attacco alle istituzioni democratiche. \\ 
Nel 2024 si è nuovamente candidato alla presidenza degli Stati Uniti affrontando una campagna elettorale caratterizzata da una retorica ancora più estrema e da un coinvolgimento senza precedenti dei suoi sostenitori \cite{nyt2024, bbc2024}. Nonostante le numerose controversie e le sfide giudiziarie è riuscito a riconquistare la leadership del Partito Repubblicano e ad imporsi nelle elezioni presidenziali, diventando così il 47º presidente degli Stati Uniti. La sua vittoria ha segnato un evento storico, rappresentando il ritorno alla Casa Bianca di un ex presidente dopo una sconfitta elettorale, e ha ulteriormente accentuato le divisioni politiche e sociali all'interno del Paese. \\
Il secondo mandato di Trump si apre in un contesto di forte tensione, con l’attenzione dei media e dell’opinione pubblica concentrata sulle sue prime decisioni e sulle possibili conseguenze per la politica interna ed estera degli Stati Uniti.\\

\section{Centralità mediatica e polarizzazione}

La centralità mediatica di Donald Trump, già evidente durante la sua prima presidenza, ha raggiunto nuovi livelli in occasione della campagna elettorale e della vittoria alle elezioni del 2024 \cite{lse2024}. Trump ha saputo sfruttare un ecosistema mediatico sempre più frammentato, caratterizzato dalla presenza di media tradizionali, piattaforme digitali, social network e, in misura crescente, podcast e canali di informazione alternativi. Questa frammentazione ha contribuito a rafforzare la polarizzazione del discorso pubblico, creando ``bolle informative'' in cui le narrazioni su Trump e sui suoi avversari risultano spesso inconciliabili \cite{pew2024}. Durante la campagna del 2024, Trump ha puntato in modo strategico su media non convenzionali, come podcast e piattaforme di streaming, raggiungendo segmenti di pubblico tradizionalmente meno coinvolti dalla politica, in particolare giovani uomini e gruppi sociali ``disillusi'' dai media mainstream. La sua presenza su podcast di grande seguito, come ``The Joe Rogan Experience'', ha permesso di bypassare i filtri giornalistici tradizionali e di consolidare un rapporto diretto e personale con l’elettorato. Questa strategia ha contribuito a un aumento significativo del consenso tra giovani e minoranze, pur restando la base elettorale di Trump prevalentemente composta da elettori bianchi \cite{newsweek2024}. La copertura mediatica della vittoria di Trump nelle ultime elezioni ha evidenziato una profonda spaccatura tra le narrazioni delle testate conservatrici e quelle progressiste. I media di destra hanno celebrato il ritorno di Trump come una rivincita contro l’establishment e i media tradizionali, mentre le testate progressiste hanno sottolineato i rischi per la democrazia e la crescente disinformazione. Questa dicotomia si riflette anche nella fiducia del pubblico nei confronti dei media: secondo recenti sondaggi, la fiducia nelle testate tradizionali ha raggiunto livelli storicamente bassi, con una crescente parte della popolazione che si informa esclusivamente tramite fonti alternative o social media \cite{pew2024}. \\
Un altro elemento centrale della polarizzazione è la tendenza dei media a enfatizzare il ruolo di specifici gruppi demografici nel successo elettorale di Trump, spesso attribuendo la vittoria a presunti ``spostamenti'' di voto tra minoranze etniche. Tuttavia, analisi più approfondite mostrano che la coalizione elettorale di Trump rimane prevalentemente bianca, e che la narrazione mediatica rischia di semplificare eccessivamente dinamiche sociali complesse, alimentando ulteriori divisioni \cite{georgetown2025}. \\
In sintesi, la rielezione di Donald Trump nel 2024 ha rappresentato non solo un evento politico di portata storica, ma anche un caso emblematico di come la centralità mediatica e la polarizzazione informativa possano influenzare profondamente la percezione pubblica e il dibattito democratico. L’analisi delle narrazioni giornalistiche relative agli eventi della sua campagna e presidenza offre uno spaccato significativo delle dinamiche di potere, fiducia e conflitto che caratterizzano il sistema mediatico contemporaneo \cite{faris2017}.